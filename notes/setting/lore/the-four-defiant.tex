\section{Shomah Thorne and the Four Defiant -- The Origin of the Empire}

\textit{This is a story told throughout the Empire, by bards in barrooms, by youths around
  campfires, and by parents (albeit less explicit) to their restless children.
While all of them -- bards, youths, and parents -- are notorious liars, the story of Shomah's
  march across the modern-day Empire is generally accepted as truth, within the bounds of
  exaggeration accepted as good story-telling.
While not going so far as to have their own temples, Shomah and The Four Defiant themselves are now
  revered almost as gods for their foundational importance to the Empire.}

\medskip

In the days before the Empire, we warred with the beastmen constantly.
Herders worried less about wolves taking sheep than they did about gnolls taking their lives.
Despair hung over the land like a heavy fog.
These dark, harsh times made dark, harsh people.
People like Shomah.

Shomah rose from the glowing embers of her ruined village with nothing but a sword and a promise:
  the beastmen would pay to their last.
The human traveled from town to town, taking the long and dangerous paths.
In her wake, the beast-tribes burned.
Gnolls lay be the roadside, gnarled and twisted.
Kobold caves filled with their fallen, each one cracked and broken.
Lizardmen lay lacerated.
Bullywug bodies bloated.

In time, Shomah's name would demand fear and reverence in equal parts.
In every city she would visit, every town, every village, every hamlet,
  men and women would join her.
They would make their own promises and take up their own weapons, for all had lost something
  to the beastmen.
Shomah would lead them, always along the long and dangerous paths, calling due every bloody debt.

Yet in those early days, before her host became her army, before her ruined home became the seat of
  an Empire and she our First Empress, before any of that, she was joined by a band
  -- the Four Defiant.
The Defiant had themselves crushed many beastmen under their collective heel,
  not for revenge but for profit.
The band of adventurers kept the area safe and themselves fed.
They were known as talented mercenaries, but altruists, they were not.

No one knows what Shomah said to bring them to her cause driving out the beastmen.
Certainly, she had no coin by which to hire them.
Perhaps she pleaded, her impassioned call fanning a hope the Defiant had themselves
  thought long smothered.
Perhaps they recognized in Shomah the last element.
With her, they would form an alloy no enemy could break.

Whatever the reason, the five left that western town of Breygrove on a long trek eastwards, toward
destiny.

\hrulefill

Shomah stands alone, her longsword held in both hands.
Three gnolls stand before her, yipping at her uncertainly.
Behind her, the bodies of other hyena mark where she has been.
Gnoll blood stains her hair and drips slowly into her eye.
The hyenamen sense the opportunity, attacking together, but even partially blinded, they are no
  match for Shomah.
She has led this dance so many times, it comes as easily as breathing.

Some distance away, Poise stands motionless in the center of a circle of the beastmen.
Her pale hair floats wild in the wind.
She holds no weapon.
She shouts no threat nor plea.
Her downcast eyes and small frame belie her threat.
She is the patient, ceaseless roll of the sea.
She is the undertow that drags the unwitting beyond the safety of shore to drown them there.

###
The Raven's laugh croaks sharply through the battlefield.
The Elf is old even among Elves.
Her skin like charcoal and ash is dusted with actual ash and smeared with sweat.
A gnoll charges at her, spear leveled.
Her tattered robes cover tattooed whorls, which whirl and roil when she work her spells.
In a flash of flame, the gnoll is gone, more ash filling the air.
The Raven's harsh laugh sounds again.

From atop a hastily-deployed palisade, a half-Orc balances precariously to watch his allies.
Soot and grease stain the Wrecker's leather gloves, leather apron, and rough leathery face.
The Wrecker could break a siege by himself, given enough time, iron, and wire.

His heavy work hammer and strong arm were deadly in a melee, but deadlier still in his workshop.









Celadir's understanding of combat let the Defiant predict every attack,
but it was often the Raven's magic that let them survive it.


Of the Four Defiant, only Celadir's name lives on.
The half-Elf would later be named Regent, though he wouldn't know it at the time
  and would come to renounce it in the end.
Shomah would hold Celadir foremost among her councilors,
  count him first of her few friends,
  and, some rumored, invite him alone to into the circle of her arms.
Throughout history both before and since, none can call themselves equal
  to Celadir's tactical genius.
It is said that for every life Shomah ended, Celadir's wisdom in battle saved dozens.

\hrulefill

And so the group traveled the roads that today mark the outskirts of our Empire.
Their name, fame, and fortune grew, and with that so too did their number.

Shomah and her warriors pressed onward.
Before their blades, the beastmen fell.
Before their wrath, the beastmen fled.

Shomah, the Defiant, and their host drove the beastmen east,
  toward the great and wild Rahlu River.
On they pressed them, to the ancient, wide bridge called the Bastion.

That night, they paused in their attack, Shomah and her forces on the western side of the Bastion,
  the dwindling number of beastmen on the east.



####
The host driven east.
Shomah pursued.
Celadir voiced concern about exposing themselves.
Shomah orders her four most trusted warriors, advisors, friends to stay.
The rest of the host pushes north.

The Yuan-ti, the serpent-men, are as devious as they are vile.
Shomah is assassinated.
The Yuan-ti descend upon the western banks of the river.
They are not prepared.

The Raven, The Wrecker, The Regent, The Wrath, score a line in the earth.
Those abominations that dared to cross it were mete a brutal reward.

The Yuan-ti horde flowed for ages, but the fervent fury of the four was unending.
