\section{The Northern Wylde}
\headeritem{Structure}{Desolation}

The farther north one travels, the more dense the forest becomes.
A small village, Dileah, is nestled in these woods, built up as a logging camp expanded outward.
This village and the roads from it mark the edge of The Empire's will and interest in the norther
  reaches.

A day's walk north of Dileah, scraps of white cloth dangle from tree limbs, another tied every
  dozen strides.
This ragged line traces roughly the line that which a cartographer would draw on a map.
Locals call this "the edge."
This marks the edge of The Empire's promise of protection.

Continue several hours farther, through thick bramble and undergrowth, and you would find
  bright red ribbons tied in a similar fashion.
These drift and snap in an unfelt wind, despite the heavy stillness of the woods.
They mark the edge of safety.

According to the folklore in Dileah and the surrounding villages, the "rag line" marks the territory
  of The Ragged Witch.
Most who cross the rag line are never seen again.
The few that return are found beaten, bloodied, gagged, and bound beneath one of the
  white scraps marking the edge.
These few will tell of a dark-skinned, gaunt figure clad in layer upon layer of tattered robes and
  cloaks.
Her voice strikes like iron, rebuking their trespass, shortly before darkness descends.
